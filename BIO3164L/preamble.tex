%
% C'est ici que sont définies les principales options pour LaTeX
%

%%%%%%%%%%%%%%%%%
% Paquets LaTeX %
%%%%%%%%%%%%%%%%%

\usepackage[french]{babel} % Pour un document en français
\usepackage{lettrine}      % Pour faire des lettrines
\usepackage{verbatim}      % Pour pouvoir inclure ce fichier tel quel
                           % dans le document final
\usepackage{xcolor} % Pour faire du texte en couleur


%%%%%%%%%%%%%%%%%%%%%%%%%%%%%%%%%%
% Définitions de macros LaTeX %
%%%%%%%%%%%%%%%%%%%%%%%%%%%%%%%%%%

% Définition des couleurs utilisées

\definecolor{bleu}{rgb}{0.0,0,0.3}
\definecolor{rouge}{rgb}{0.3,0.0,0.0}
\definecolor{noir}{rgb}{0.0,0.0,0.0}

% Pour faire référence à la section trucmuche page machin, par
% exemple "section 1.2.3 page 18". Le seul argument est le
% nom du label associé à la section. Le caractère ~ représente
% une espace insécable.

\newcommand{\refs}[1]{section~\ref{#1} page~\pageref{#1}}

% Pour faire référence à la figure trucmuche page machin

\newcommand{\reff}[1]{figure~\ref{#1} page~\pageref{#1}}

% Pour faire référence à la table trucmuche page machin

\newcommand{\reft}[1]{table~\ref{#1} page~\pageref{#1}}

%
% Pour insérer une figure au format jpg dans le dossier figs.
% Le premier argument est le nom du fichier sans extension,
% le deuxième la taille horizontale de la figure, le troisième
% la légende de la figure. Un label du même nom que le fichier
% est inséré pour pouvoir faire facilement référence à la figure.
%

\newcommand{\myjpg}[3]{
  \begin{figure}
    \begin{center}
      \fbox{\begin{minipage}{#2\textwidth}
        \includegraphics[width=\textwidth]{figs/#1.jpg}
        \caption{\label{#1} #3}
      \end{minipage}}
    \end{center}
  \end{figure}
}